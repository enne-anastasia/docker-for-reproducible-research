% Options for packages loaded elsewhere
\PassOptionsToPackage{unicode}{hyperref}
\PassOptionsToPackage{hyphens}{url}
\documentclass[
  ignorenonframetext,
  aspectratio=169]{beamer}
\newif\ifbibliography
\usepackage{pgfpages}
\setbeamertemplate{caption}[numbered]
\setbeamertemplate{caption label separator}{: }
\setbeamercolor{caption name}{fg=normal text.fg}
\beamertemplatenavigationsymbolsempty
% remove section numbering
\setbeamertemplate{part page}{
  \centering
  \begin{beamercolorbox}[sep=16pt,center]{part title}
    \usebeamerfont{part title}\insertpart\par
  \end{beamercolorbox}
}
\setbeamertemplate{section page}{
  \centering
  \begin{beamercolorbox}[sep=12pt,center]{section title}
    \usebeamerfont{section title}\insertsection\par
  \end{beamercolorbox}
}
\setbeamertemplate{subsection page}{
  \centering
  \begin{beamercolorbox}[sep=8pt,center]{subsection title}
    \usebeamerfont{subsection title}\insertsubsection\par
  \end{beamercolorbox}
}
% Prevent slide breaks in the middle of a paragraph
\widowpenalties 1 10000
\raggedbottom
\AtBeginPart{
  \frame{\partpage}
}
\AtBeginSection{
  \ifbibliography
  \else
    \frame{\sectionpage}
  \fi
}
\AtBeginSubsection{
  \frame{\subsectionpage}
}
\usepackage{iftex}
\ifPDFTeX
  \usepackage[T1]{fontenc}
  \usepackage[utf8]{inputenc}
  \usepackage{textcomp} % provide euro and other symbols
\else % if luatex or xetex
  \usepackage{unicode-math} % this also loads fontspec
  \defaultfontfeatures{Scale=MatchLowercase}
  \defaultfontfeatures[\rmfamily]{Ligatures=TeX,Scale=1}
\fi
\usepackage{lmodern}
\usetheme[]{CambridgeUS}
\usefonttheme[]{structurebold}
\ifPDFTeX\else
  % xetex/luatex font selection
\fi
% Use upquote if available, for straight quotes in verbatim environments
\IfFileExists{upquote.sty}{\usepackage{upquote}}{}
\IfFileExists{microtype.sty}{% use microtype if available
  \usepackage[]{microtype}
  \UseMicrotypeSet[protrusion]{basicmath} % disable protrusion for tt fonts
}{}
\makeatletter
\@ifundefined{KOMAClassName}{% if non-KOMA class
  \IfFileExists{parskip.sty}{%
    \usepackage{parskip}
  }{% else
    \setlength{\parindent}{0pt}
    \setlength{\parskip}{6pt plus 2pt minus 1pt}}
}{% if KOMA class
  \KOMAoptions{parskip=half}}
\makeatother
\setlength{\emergencystretch}{3em} % prevent overfull lines
\providecommand{\tightlist}{%
  \setlength{\itemsep}{0pt}\setlength{\parskip}{0pt}}
\usepackage[style=authoryear,]{biblatex}
\addbibresource{bibliography.bib}
\usepackage{float}
\usepackage[para]{footmisc} % multiple footnotes in one line
\renewcommand{\footnotesize}{\tiny}
\usepackage{bookmark}
\IfFileExists{xurl.sty}{\usepackage{xurl}}{} % add URL line breaks if available
\urlstyle{same}
\hypersetup{
  pdftitle={Docker for Reproducible Research},
  pdfauthor={Anastasiia Enne},
  hidelinks,
  pdfcreator={LaTeX via pandoc}}

\title{Docker for Reproducible Research}
\author{Anastasiia Enne}
\date{Group meeting, 14.01.2026}

\begin{document}
\frame{\titlepage}

\begin{frame}
\frametitle{Overview}

\begin{enumerate}
  \item \textbf{Reproducibility crisis} --- why just publishing your code is not enough
  \item \textbf{What is Docker} and how can it help us with reproducibility
  \item \textbf{Dockerizing an existing project} --- a step-by-step guide if you want to try Docker
\end{enumerate}
\end{frame}

\begin{frame}
\frametitle{Reproducibility crisis}

\begin{columns}

  \begin{column}{0.45\textwidth}
    \begin{itemize}
      \item<1-> Reminder: 4 definitions of reproducible research\footnotemark[1]
      \item<2-> Our focus will be on REPRODUCIBILITY
      \item<3-> Most scientists report that they have failed to reproduce an experiment\footnotemark[2]
      \item<4-> But what about modelling?
    \end{itemize}
  \end{column}
  
  \begin{column}{0.45\textwidth}
    \begin{overprint}
      \centering
      \onslide<1>\includegraphics[width=\linewidth]{../figures/downloaded/the-turing-way/reproducible-definitiongrid.jpg}
      \onslide<2>\includegraphics[width=\linewidth]{../figures/downloaded/the-turing-way/reproducible-definitiongrid-reproducibility-highlighted.jpg}
      \onslide<3->\includegraphics[width=0.8\linewidth]{../figures/downloaded/papers/baker_1500_2016.png}
    \end{overprint}
  \end{column}
  
  \only<1->{\footnotetext[1]{The Turing Way Community. This illustration is created by Scriberia with The Turing Way community, used under a CC-BY 4.0 licence. DOI: \textcolor{magenta}{\href{https://doi.org/10.5281/zenodo.3332807}{10.5281/zenodo.3332807}}}}
  \only<3->{\footnotetext[2]{\parencite{baker_1500_2016}}}
\end{columns}
\end{frame}

\begin{frame}
\frametitle{Reproducibility crisis in modelling studies?}

\begin{columns}
  
  \begin{column}{0.45\textwidth}
    \begin{itemize}
      \item<1-> \textit{How often do you manage to find and run the code for a paper without issues?}
      \item<2-> Not many studies have investigated reproducibility in modelling papers...
      \begin{itemize}
        \item<3-> Out of 7500 papers about individual-based and agent-based models only 11.2\% provided their code\footnotemark[1]
        \item<4-> Out of 455 ODE models from \textcolor{magenta}{\href{https://www.biomodels.org/}{www.biomodels.org}} 49\% are not directly reproducible\footnotemark[2]
      \end{itemize}
    \end{itemize}
  \end{column}
  
  \begin{column}{0.45\textwidth}
    \begin{overprint}
      \centering
      \onslide<3>\includegraphics[width=0.9\textwidth]{../figures/downloaded/papers/janssen_code_2020_figure_2.jpg}
      \onslide<4>\includegraphics[width=0.9\textwidth]{../figures/downloaded/papers/tiwari_reproducibility_2021_figure_1.png}
    \end{overprint}
  \end{column}
  
  \only<3->{\footnotetext[1]{\parencite{janssen_code_2020}}}
  
  \only<4->{\footnotetext[2]{\parencite{tiwari_reproducibility_2021}}}

\end{columns}
\end{frame}

\begin{frame}
\frametitle{Reproducibility crisis in modelling studies!}

\begin{columns}

  \begin{column}{0.45\textwidth}
    Main barriers to reproducibility\footnotemark[1]:
    \begin{itemize}
      \item<1-> \textbf{"Dependency Hell"}: it might be difficult to recreate the original computational environment
      \item<2-> \textbf{Imprecise documentation}: hard to install and build the code, without all parameter values impossible to recreate the results 
      \item<3-> \textbf{Code rot}: software dependencies get updated all the time, so without proper maintenance the code will become unrunable at some point
    \end{itemize}
  \end{column}
  
  \begin{column}{0.45\textwidth}
    \onslide<4->{What could help us solve these problems?\footnotemark[1]}
    \begin{itemize}
      \item<5-> \textbf{Virtual machines}
      \begin{itemize}
        \item<5-> Computationally heavy and not scalable (hard to combine code from multiple studies)
        \item<5-> "Black box" without clear list of dependencies
      \end{itemize}
      \item<6-> \textbf{Workflows}
      \begin{itemize}
        \item<6-> A lot of proprietary formats
        \item<6-> Limited functionality
      \end{itemize}
      \item<7-> \textbf{Docker}
      \begin{itemize}
        \item<7-> Lightweight and easily scalable
        \item<7-> Clear list of dependencies
        \item<7-> Open source
        \item<7-> Linux functionality
      \end{itemize}
    \end{itemize}
  \end{column}

\footnotetext[1]{\parencite{boettiger_introduction_2015}}
  
\end{columns}
\end{frame}

\begin{frame}
\frametitle{Learning objectives}

\begin{columns}

  \begin{column}{0.45\textwidth}
    \begin{itemize}
      \item<1-> Docker becomes the new standard for reproducibility in science (e.g. Methods in Ecology and Evolution require the code to be Dockerized).
      \item<2-> What should ideally be achieved by the end of the talk:
        \begin{itemize}
          \item<2-> \textbf{Understanding the basics of Docker and its components}
          \item<2-> Being able to to create and manage Docker containers
        \end{itemize}
      \item<3-> What is not going to be covered:
        \begin{itemize}
          \item<3-> Docker automations for R (see \cite{chan_rang_2023})
          \item<3-> Docker desktop app
        \end{itemize}
    \end{itemize}
  \end{column}
  
  \begin{column}{0.45\textwidth}
    \centering
    \onslide<1->\includegraphics[width=0.7\textwidth]{../figures/downloaded/Docker/Docker_(container_engine)_logo_(cropped).png}
    
    \onslide<4->{These slides are baked with Rmd and Dockerized! Go to\\
    \textcolor{magenta}{\href{https://github.com/enne-anastasia/docker-for-reproducible-research}{github.com/enne-anastasia/docker-for-reproducible-research}} and see how I used Docker!}
  \end{column}

\end{columns}
\end{frame}

\begin{frame}
\frametitle{What is Docker?}

\begin{block}{Main idea}
  \begin{itemize}
    \item<1-> Alice makes a code supplementary that Bob wants to reproduce.
    \item<2-> With her supplementary Alice provides a small text file, that is both human-readable and machine-readable, and documents all the dependencies, as well as steps one needs to take in order to reproduce her analysis.
    \item<3-> Bob can use Docker software to build an exact copy of Alice's environment from this text file, that will act like a virtual machine, but use much less resources.
    \item<4-> If Bob doesn't want to be bothered with using Docker he still can read the file and install all the dependencies on his own device manually.
    \item<5-> Alice is certain that she provided all the dependencies, because she tested this file with the Docker software.
  \end{itemize}
\end{block}
\end{frame}

\begin{frame}
\frametitle{What is Docker?}

\begin{columns}

  \begin{column}{0.45\textwidth}
    \hspace{3cm}\texttt{docker build} $\Rightarrow$
  \end{column}
  
  \begin{column}{0.45\textwidth}
    \hspace{1cm}\texttt{docker run} $\Rightarrow$
  \end{column}

\end{columns}

\hfill

\begin{columns}

  \begin{column}{0.3\textwidth}
    \centering
    \includegraphics[width=0.3\textwidth]{../figures/downloaded/Docker/docker_file.png}
    
    \textbf{Docker file}
  \end{column}
  
  \begin{column}{0.3\textwidth}
    \centering
    \includegraphics[width=0.3\textwidth]{../figures/downloaded/Docker/docker_image.png}
    
    \textbf{Docker image}
  \end{column}
  
  \begin{column}{0.3\textwidth}
    \centering
    \includegraphics[width=0.3\textwidth]{../figures/downloaded/Docker/docker_container.png}
    
    \textbf{Docker container}
  \end{column}

\end{columns}

\hfill

\begin{columns}

  \begin{column}{0.3\textwidth}
    \begin{itemize}
      \item Text file with "source code" of the image:
      \begin{itemize}
        \item Instructions on how to build the image
        \item Commands to run your project
      \end{itemize}
    \end{itemize}
  \end{column}
  
  \begin{column}{0.3\textwidth}
    \begin{itemize}
      \item An executable snapshot of a container
      \item Includes all dependencies needed to run a container
    \end{itemize}
  \end{column}
  
  \begin{column}{0.3\textwidth}
    \begin{itemize}
      \item A running instance of the image
      \item Your "pocket Ubuntu"
      \item Self-contained
    \end{itemize}
  \end{column}

\end{columns}
\end{frame}

\begin{frame}
\frametitle{What is Docker?}
\framesubtitle{More about Docker containers}

\begin{columns}

  \begin{column}{0.45\textwidth}
    \centering
    \includegraphics[width=0.3\textwidth]{../figures/downloaded/Docker/docker_container.png}
    
    \textbf{Docker container}
    
    \hfill
    
    \begin{itemize}
      \item<1-> Containers stop existing after you exit them, meaning that all changes within the containers are lost.
      \item<1-> Containers are isolated from your file systems.
    \end{itemize}
  \end{column}
  
  \begin{column}{0.45\textwidth}
    \begin{itemize}
      \item<2-> Consequently, you need to either
      \begin{itemize}
        \item<2-> copy your files into the image (with a command in your Docker file), or
        \item<2-> mount a folder to a container (during \texttt{docker run})
      \end{itemize}
      \item<3-> The rule of thumb is:
      \begin{itemize}
        \item<3-> We MOUNT everything that takes plenty of space: e.g., data
        \item<3-> We MOUNT everything that we want to be changed: e.g., figures
        \item<3-> We COPY source code
      \end{itemize}
    \end{itemize}
  \end{column}

\end{columns}
\end{frame}

\begin{frame}
\frametitle{Dockerizing an existing project}
\framesubtitle{What project are we going to Dockerize?}

\begin{columns}

  \begin{column}{0.45\textwidth}
    \begin{itemize}
      \item We will Dockerize the R project that I used to prepare this presentation
      \item It has a very common structure for a code supplementary:
      \begin{itemize}
        \item \texttt{data} --- folder containing all the data used in this project (empty in this case)
        \item \texttt{docs} --- folder with \texttt{Rmd} file that generates these slides
        \item \texttt{figures} --- folder with all the figures I used
        \item \texttt{README} file
        \item \texttt{run\_analysis.sh} --- one script to execute this project
        \item \texttt{src} --- source code (1 test R script in this case)
      \end{itemize}
    \end{itemize}
  \end{column}
  
  \begin{column}{0.5\textwidth}
    \begin{overprint}
      \centering
      \onslide<1>\includegraphics[width=1.2\textwidth]{../figures/screenshots/folder_structure_initial.png}
    \end{overprint}
  \end{column}

\end{columns}
\end{frame}

\begin{frame}
\frametitle{Dockerizing an existing project}
\framesubtitle{STEP 1: Installing Docker}

\begin{columns}

  \begin{column}{0.45\textwidth}
    \begin{itemize}
      \item<1-> The official \textcolor{magenta}{\href{https://docs.docker.com/}{docs.docker.com}} provides full manuals on how to install Docker:
      \begin{itemize}
        \item<1-> To work from the command line on Linux \textcolor{magenta}{\href{https://docs.docker.com/engine/install/}{install Docker Engine}} (\textbf{this is what I did} on \textcolor{magenta}{\href{https://docs.docker.com/engine/install/ubuntu/}{Ubuntu}})
        \item<2-> To work with UI \textcolor{magenta}{\href{https://docs.docker.com/desktop/}{install Docker Desktop}} for \textcolor{magenta}{\href{https://docs.docker.com/desktop/setup/install/linux/}{Linux}}, \textcolor{magenta}{\href{https://docs.docker.com/desktop/setup/install/windows-install/}{Windows}}, or \textcolor{magenta}{\href{https://docs.docker.com/desktop/setup/install/mac-install/}{Mac}} (beware, this is less "light-weight" compared to Docker in command line)
      \end{itemize}
      \item<3-> Verify that the installation is successful:\\
      \texttt{sudo systemctl status docker}\\
      \texttt{sudo docker run hello-world}
    \end{itemize}
  \end{column}
  
  \begin{column}{0.5\textwidth}
    \begin{overprint}
      \centering
      \onslide<3>\includegraphics[width=\textwidth]{../figures/screenshots/status_docker.png}
      \onslide<4>\includegraphics[width=\textwidth]{../figures/screenshots/docker_run_hello-world.png}
    \end{overprint}
  \end{column}

\end{columns}
\end{frame}

\begin{frame}
\frametitle{Dockerizing an existing project}
\framesubtitle{STEP 2: Trying the basic Docker commands}

\begin{columns}

  \begin{column}{0.45\textwidth}
    \begin{itemize}
      \item<1,2> \textbf{Starting Docker:}\\
      \texttt{sudo systemctl start docker}\\
      \textit{\small (Docker is running in the background waiting for you to use it)}
      \item<1,3,4> \textbf{Stopping Docker:}\\
      \texttt{sudo systemctl stop docker}
      \textit{\small (Docker is deactivated, but sometimes \texttt{docker.socket} remains active)}
      \item<1,2,3,4> \textbf{Checking Docker status:}\\
      \texttt{sudo systemctl status docker}
      \item<4> To kill \texttt{docker.socket} use\\
      \texttt{sudo systemctl stop docker.socket}
    \end{itemize}
  \end{column}
  
  \begin{column}{0.5\textwidth}
    \begin{overprint}
      \centering
      \onslide<2>\includegraphics[width=\textwidth]{../figures/screenshots/status_docker.png}
      \onslide<3>\includegraphics[width=\textwidth]{../figures/screenshots/status_docker_stopped.png}
      \onslide<4>\includegraphics[width=\textwidth]{../figures/screenshots/status_docker_stopped_no_socket.png}
    \end{overprint}
  \end{column}

\end{columns}
\end{frame}

\begin{frame}
\frametitle{Dockerizing an existing project}
\framesubtitle{STEP 2: Trying the basic Docker commands}

\begin{columns}

  \begin{column}{0.5\textwidth}
    \begin{itemize}
      \item<1-> \textbf{Listing all available Docker images:}\\
      \texttt{sudo docker image ls}
      \item<2-> \textbf{Listing all containers that are currently running:}\\
      \texttt{sudo docker container ls -a}
      \item<3-> \textbf{Deleting a container:}\\
      \texttt{sudo docker container rm <ID>}\\
      \texttt{sudo docker container rm <NAME>}
      \item<4-> \textbf{Deleting an image:}\\
      \texttt{sudo docker image rm <ID>}\\
      \texttt{sudo docker image rm <IMAGE>}
    \end{itemize}
  \end{column}
  
  \begin{column}{0.5\textwidth}
    \begin{overprint}
      \centering
      \onslide<1,4>\includegraphics[width=\textwidth]{../figures/screenshots/docker_image_ls.png}
      \onslide<2,3>\includegraphics[width=\textwidth]{../figures/screenshots/docker_container_ls_-a.png}
    \end{overprint}
  \end{column}

\end{columns}
\end{frame}

\begin{frame}
\frametitle{Dockerizing an existing project}
\framesubtitle{STEP 3: Writing the Docker file}

\begin{columns}

  \begin{column}{0.45\textwidth}
    \begin{enumerate}
      \item<1-> Create an empty \texttt{Dockerfile} in the root directory of you project
      \item<2-> Pick the base image
      \begin{itemize}
        \item<2-> Writing your Docker file from scratch is like setting a new work laptop from scratch
        \item<2-> Thankfully, we do not have to do that because there are Docker images we can build upon!
        \item<3-> For this project I will use \texttt{rocker} --- \textcolor{magenta}{\href{https://rocker-project.org/}{Docker image with pre-installed R}}
      \end{itemize}
    \end{enumerate}
  \end{column}
  
  \begin{column}{0.5\textwidth}
    \begin{overprint}
      \centering
      \onslide<1,2>\includegraphics[width=1.2\textwidth]{../figures/screenshots/folder_structure.png}
      \onslide<3>\includegraphics[width=\textwidth]{../figures/screenshots/Rocker.png}
    \end{overprint}
  \end{column}

\end{columns}
\end{frame}

\begin{frame}
\frametitle{Dockerizing an existing project}
\framesubtitle{STEP 3: Writing the Docker file}

\begin{columns}

  \begin{column}{0.45\textwidth}
    \begin{enumerate}
      \setcounter{enumi}{2}
      \item<1-> Write the first line in your \texttt{Dockerfile}:
      \begin{itemize}
        \item<1-> We build our image starting from Docker image with pre-installed R
        \item<1-> I specify the version 4.5.2 of R since this is the version that I currently have on my laptop
        \item<1-> There can be only one \texttt{FROM} image!
      \end{itemize}
    \end{enumerate}
    \begin{itemize}
      \item<2-> Congratulations! This is the the minimal Docker file we can use to build and run a Docker image (atlhough, not very usefull one)
    \end{itemize}
  \end{column}
  
  \begin{column}{0.5\textwidth}
    \begin{overprint}
      \centering
      \onslide<1->\includegraphics[width=\textwidth]{../figures/screenshots/Dockerfile_FROM.png}
    \end{overprint}
  \end{column}

\end{columns}
\end{frame}

\begin{frame}
\frametitle{Dockerizing an existing project}
\framesubtitle{STEP 3: Writing the Docker file}

\begin{columns}

  \begin{column}{0.45\textwidth}
    \begin{enumerate}
      \setcounter{enumi}{3}
      \item Specify software dependencies
        \begin{itemize}
          \item It is okay to do that in iterations. When you think you got all the dependencies, go to the next step and see if you can run your project in the Docker container.
          \item Here I have 2 types of dependencies: OS-level and R-level.
        \end{itemize}
    \end{enumerate}
  \end{column}
  
  \begin{column}{0.5\textwidth}
    \begin{overprint}
      \centering
      \onslide<1->\includegraphics[width=\textwidth]{../figures/screenshots/Dockerfile_dependencies.png}
    \end{overprint}
  \end{column}

\end{columns}
\end{frame}

\begin{frame}
\frametitle{Dockerizing an existing project}
\framesubtitle{STEP 3: Writing the Docker file}

\begin{columns}

  \begin{column}{0.45\textwidth}
    \begin{enumerate}
      \setcounter{enumi}{4}
      \item<1-> Copy the source code into the image
      \begin{itemize}
        \item<1-> The inside of the container is an Ubuntu file system with installed dependencies, therefore I copy everything into a separate folder to keep things neat.
      \end{itemize}
      \item<2-> Write the last line with the command to execute the analysis
      \begin{itemize}
        \item<2-> There can be only one \texttt{CMD} line in your Docker file!
      \end{itemize}
    \end{enumerate}
  \end{column}
  
  \begin{column}{0.5\textwidth}
    \begin{overprint}
      \centering
      \onslide<1>\includegraphics[width=\textwidth]{../figures/screenshots/Dockerfile_COPY.png}
      \onslide<2>\includegraphics[width=\textwidth]{../figures/screenshots/Dockerfile_CMD.png}
      \onslide<3>\includegraphics[width=\textwidth]{../figures/screenshots/run_analysis.png}
    \end{overprint}
  \end{column}

\end{columns}
\end{frame}

\begin{frame}
\frametitle{Dockerizing an existing project}
\framesubtitle{STEP 4: Building the image}

\begin{columns}

  \begin{column}{0.5\textwidth}
    \texttt{sudo docker build -t <TAG\_NAME> ./}
    \begin{itemize}
      \item It probably will take a while to install all dependencies
      \item When you change one layer of your image and build again, Docker will execute only starting from layers that were changed
    \end{itemize}
  \end{column}
  
  \begin{column}{0.5\textwidth}
    \begin{overprint}
      \centering
      \onslide<1>\includegraphics[width=\textwidth]{../figures/screenshots/docker_build.png}
      \onslide<2>\includegraphics[width=\textwidth]{../figures/screenshots/docker_build_image.png}
    \end{overprint}
  \end{column}

\end{columns}
\end{frame}

\begin{frame}
\frametitle{Dockerizing an existing project}
\framesubtitle{STEP 5: Testing your container}

\texttt{sudo docker run -v \$(pwd)/figures:/docker\_reproducibility/figures -v \$(pwd)/docs:/docker\_reproducibility/docs <TAG\_NAME>}
\end{frame}

\begin{frame}
\frametitle{Dockerizing an existing project}
\framesubtitle{STEP 5: Testing your container}

\begin{overprint}
    \centering
    \onslide<1>\includegraphics[width=0.5\textwidth]{../figures/screenshots/docker_run.png}
  \end{overprint}
\end{frame}

\begin{frame}[allowframebreaks]{}
  \bibliographytrue
  \printbibliography[heading=none]
\end{frame}

\end{document}
